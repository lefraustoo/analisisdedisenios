\documentclass[12pt]{article}

% Paquetes necesarios
\usepackage[utf8]{inputenc}
\usepackage[spanish]{babel}
\usepackage{geometry}
\usepackage{graphicx}
\usepackage{amsmath}
\usepackage{hyperref}
\usepackage{bookmark}
\usepackage{rerunfilecheck}
\geometry{margin=1in}

\title{\textbf{Estudio de Viabilidad y Análisis de Requerimientos para el Sistema de Gestión de Maquinaria en Taller Mecánico Universitario}}
\author{Cortes Reyes Yarely Lizette  \\ Esquivel Velázquez Emmanuel \\ Frausto Orozco Leonardo Eldriel \\ Marcos Ruíz Emiliano \\ 1MM7}
\date{\today}

\begin{document}

\maketitle

\tableofcontents
\newpage

% Introducción
\section{Introducción}
Este documento presenta el estudio de viabilidad y el análisis de requerimientos para el desarrollo de un sistema de gestión de maquinaria en un taller mecánico universitario. El propósito es optimizar el uso y la reserva de equipo, mejorando la eficiencia y disponibilidad de las máquinas para los estudiantes.

\section{Estudio de Viabilidad}

\subsection{1. Alcance del Sistema}
Este sistema permitirá gestionar el uso de diversas máquinas en el taller (fresadora, torno, máquinas cnc, estación de soldaura, simuladores y dobladora), proporcionando un mecanismo de reserva, control de disponibilidad y mantenimiento. Además, facilitará la asignación de máquinas basadas en la prioridad de proyectos de los estudiantes.

\subsection{2. Situación Actual}
En la situación actual, la reserva de máquinas es presencial, lo que genera conflictos y tiempos muertos entre los diferentes estudiantes que lo ocupan. La falta de un sistema de gestión reduce la eficiencia y la disponibilidad de las máquinas. Además, no se cuenta con un mecanismo para controlar el estado de las máquinas y su mantenimiento. De igual forma, algunas de las máquinas no están en funcionamiento lo cual reduce la capacidad del taller.

\subsection{3. Definición de Requisitos del Sistema}
\begin{itemize}
    \item Gestión de reservas para estudiantes.
    \item Control de estado de las máquinas (disponible, en uso, en mantenimiento).
    \item Reporte de uso y estadísticas (tiempo de uso, máquinas más solicitadas, cuantas veces ha sido reparada).
    \item Asignación priorizada de máquinas (alumnado general, trabajos terminales, visitantes).
\end{itemize}

\subsection{4. Alternativas de Solución}
\begin{itemize}
    \item \textbf{Desarrollo de un sistema de información desarrollado en lenguaje orientado a objetos.}
    \item Implementación de un sistema web.
    \item Uso de una aplicación móvil para reservas remotas.
    \item Google forms para reservas.
    \item Google calendar para control de disponibilidad.
\end{itemize}

\subsection{5. Valoración de Alternativas}
Cada alternativa será evaluada en función de su facilidad y tiempo de desarrollo, costo y facilidad de uso. La opción seleccionada deberá ser adaptable y capaz de cubrir las necesidades del taller.

\subsection{6. Selección de la Solución}
La solución seleccionada es el desarrollo de una aplicación en C++, debido a su flexibilidad, bajo costo y facilidad de implementación. Además, se ajusta a las necesidades del taller y permite una integración con el sistema de control de acceso existente. 



\newpage
\section{Informe de Viabilidad}
Se presenta un resumen de las etapas anteriores para justificar la viabilidad del sistema propuesto en términos técnicos, económicos y operativos.

% Análisis de Requerimientos
\section{Obtención y Análisis de Requerimientos}

\subsection{1. Requerimientos Identificados}
Se han identificado los siguientes requerimientos, tanto funcionales como no funcionales:
\begin{itemize}
    \item Requerimientos funcionales: reserva de máquinas, control de estados.
    \item Requerimientos no funcionales: seguridad en el acceso, interfaz intuitiva.
\end{itemize}

\subsection{2. Clasificación y Organización de Requerimientos}
Los requerimientos se organizan en categorías:
\begin{itemize}
    \item \textbf{Gestión de Reservas}: permite a los estudiantes reservar máquinas.
    \item \textbf{Gestión de Mantenimiento}: permite bloquear las máquinas para mantenimiento programado o reparaciones (no colocarla en la interfaz de reservas).
    \item \textbf{Reportes y Estadísticas}: permite generar reportes de uso y estadísticas de las máquinas (solo administradores).
    \item \textbf{Asignación Priorizada}: permite asignar máquinas basadas en la prioridad de proyectos de los estudiantes.
\end{itemize}

\subsection{3. Ordenación de Prioridades y Negociación}
La gestión de reservas y el control de estados son los requerimientos de mayor prioridad. Los reportes de uso y estadísticas son secundarios y podrán añadirse en versiones futuras. La asignación priorizada es un requerimiento opcional y dependerá de la demanda de los usuarios.

\subsection{4. Documentación de Requerimientos}
\begin{itemize}
    \item \textbf{Requerimiento 1}: El sistema debe permitir la reserva de máquinas con al menos 24 horas de anticipación y un máximo de 2 horas por estudiante con un intervalo de 4 horas entre reservas.
    \item \textbf{Requerimiento 2}: El sistema debe permitir bloquear máquinas para mantenimiento programado.
    \item \textbf{Requerimiento 3}: El sistema debe generar reportes de uso y estadísticas de las máquinas para los administradores.
\end{itemize}

\end{document}
