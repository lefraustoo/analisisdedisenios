\documentclass[12pt]{article}

% Paquetes necesarios
\usepackage[utf8]{inputenc}
\usepackage{geometry}
\usepackage{graphicx}
\usepackage{amsmath}
\usepackage{hyperref}
\usepackage{bookmark}
\usepackage{rerunfilecheck}
\geometry{margin=1in}

\title{Estudio de Viabilidad y Análisis de Requerimientos para el Sistema de Gestión de Maquinaria en Taller Mecánico Universitario}
\author{lefraustoo}
\date{\today}

\begin{document}

\maketitle

\tableofcontents
\newpage

% Introducción
\section{Introducción}
Este documento presenta el estudio de viabilidad y el análisis de requerimientos para el desarrollo de un sistema de gestión de maquinaria en un taller mecánico universitario. El propósito es optimizar el uso y la reserva de equipo, mejorando la eficiencia y disponibilidad de las máquinas para los estudiantes.

\section{Estudio de Viabilidad}

\subsection{1. Alcance del Sistema}
Este sistema permitirá gestionar el uso de diversas máquinas en el taller, proporcionando un mecanismo de reserva, control de disponibilidad y mantenimiento. Además, facilitará la asignación de máquinas basadas en la prioridad de proyectos de los estudiantes.

\subsection{2. Situación Actual}
En la situación actual, la reserva de máquinas es manual, lo que genera conflictos y tiempos muertos. La falta de un sistema de gestión reduce la eficiencia y la disponibilidad de las máquinas.

\subsection{3. Definición de Requisitos del Sistema}
\begin{itemize}
    \item Gestión de reservas para estudiantes.
    \item Control de estado de las máquinas (disponible, en uso, en mantenimiento).
    \item Reporte de uso y estadísticas.
    \item Asignación priorizada de máquinas.
\end{itemize}

\subsection{4. Alternativas de Solución}
\begin{itemize}
    \item Desarrollo de una aplicación de escritorio en C++.
    \item Implementación de un sistema web.
    \item Uso de una aplicación móvil para reservas remotas.
\end{itemize}

\subsection{5. Valoración de Alternativas}
Cada alternativa será evaluada en función de su facilidad de uso, tiempo de desarrollo y costo. La opción seleccionada deberá ser adaptable y capaz de cubrir las necesidades del taller.

\subsection{6. Selección de la Solución}
La solución seleccionada es el desarrollo de una aplicación de escritorio en C++, debido a su facilidad de integración en el taller y al cumplimiento de todos los requisitos funcionales.

\newpage
\section{Informe de Viabilidad}
Se presenta un resumen de las etapas anteriores para justificar la viabilidad del sistema propuesto en términos técnicos, económicos y operativos.

% Análisis de Requerimientos
\section{Obtención y Análisis de Requerimientos}

\subsection{1. Descubrimiento de Requerimientos}
Se han identificado los siguientes requerimientos, tanto funcionales como no funcionales:
\begin{itemize}
    \item Requerimientos funcionales: reserva de máquinas, control de estados.
    \item Requerimientos no funcionales: seguridad en el acceso, interfaz intuitiva.
\end{itemize}

\subsection{2. Clasificación y Organización de Requerimientos}
Los requerimientos se organizan en categorías:
\begin{itemize}
    \item \textbf{Gestión de Reservas}: permite a los estudiantes reservar máquinas.
    \item \textbf{Gestión de Mantenimiento}: permite bloquear máquinas para mantenimiento.
\end{itemize}

\subsection{3. Ordenación de Prioridades y Negociación}
La gestión de reservas y el control de estados son los requerimientos de mayor prioridad. Los reportes de uso y estadísticas son secundarios y podrán añadirse en versiones futuras.

\subsection{4. Documentación de Requerimientos}
\begin{itemize}
    \item \textbf{Requerimiento 1}: El sistema debe permitir la reserva de máquinas.
    \item \textbf{Requerimiento 2}: El sistema debe permitir bloquear máquinas para mantenimiento programado.
\end{itemize}

% Conclusión
\section{Conclusión}
En conclusión, el estudio de viabilidad respalda el desarrollo de un sistema de gestión de maquinaria en C++ para el taller universitario. El análisis de requerimientos permite establecer una base sólida para el desarrollo del sistema y asegura que cumple con las necesidades del taller.

\end{document}
